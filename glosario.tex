\newglossaryentry{agil}{name={Metodologías ágiles}, text={ágil}, description={Las metodologías ágiles surgieron como contraposición a las metodologías tradicionales. Son aquellas que se basan en el desarrollo iterativo y donde los requisitos y soluciones evolucionan a lo largo del tiempo según lo necesite el proyecto}}
\newglossaryentry{bullethell}{name={Bullet hell}, text={bullet hell}, description={Traducido al español como "`infierno de balas"', bullet hell es un subgénero de videojuegos que se caracteriza por llenar el entorno ("`la pantalla"') de peligros (tradicionalmente balas, de ahí su nombre) que matan, dañan o provocan algún tipo de alteración al personaje}}
%TODO Revisar: Identificar más entradas
\newglossaryentry{distribuidora}{name={Distribuidora}, text={distribuidora}, description={En el mundo de los videojuegos es una empresa que distribuye juegos, ya sean propios o de terceros, y que normalmente se encarga también de aspectos propios de una editora, como de llevar a cabo la promoción de la obra}}
%TODO Revisar: Identificar más entradas
\newglossaryentry{escenario}{name={Escenario}, text={escenario}, description={Escena o escenario es la manera en la que se suele denominar un mapa o escenario de un videojuego (en inglés, se suele denominar "`scenario"')}}
\newglossaryentry{gameplay}{name={Gameplay}, text={gameplay}, description={Forma en la que se juega a un videojuego, incluyendo las reglas, los objetivos y la forma de alcanzarlos}}
\newglossaryentry{gaming}{name={Gaming}, text={gaming}, description={Hace referencia a que el dispositivo está diseñado o enfocado para el uso relacionado con videojuegos}}
\newglossaryentry{independiente}{name={Independiente / indie}, text={independiente}, description={Generalmente se denomina videojuego  independiente o indie a aquel realizado por un número reducido de personas y/o con poco presupuesto, sin el respaldo de una gran compañía}}
\newglossaryentry{jugabilidad}{name={Jugabilidad}, text={jugabilidad}, description={Experiencia del jugador durante la interacción con las distintas \glspl{mecanica}, controles y situaciones del videojuego}}
\newglossaryentry{landscape}{name={Landscape}, text={landscape}, description={En un videojuego, un landscape es un plano, generalmente con diferentes alturas, que se utiliza principalmente para la recreación de un paisaje natural pudiendo contener montañas, valles, etc. Principalmente se usa como base para un \gls{escenario} y en él se emplazarán árboles, edificios y \glspl{modelado} de todo tipo}}
\newglossaryentry{mecanica}{name={Mecánica}, text={mecánica}, description={Son cada una de las reglas del universo del videojuego y conjunto de acciones que puede realizar el jugador. Por ejemplo: Una mecánica sería que el personaje muriese al caer desde una determinada altura o bien que el personaje, simplemente, pueda correr o saltar}}
\newglossaryentry{modelado}{name={Modelado}, text={modelado}, description={Un modelado o malla (del inglés “mesh”) es la representación mediante polígonos de un determinado objeto, que normalmente tiene asociado un material o textura, pudiendo también tener asociado varios}}
\newglossaryentry{motorgrafico}{name={Motor gráfico}, text={motor gráfico}, description={Es un framework diseñado para desarrollar videojuegos que le ofrece al programador al menos las características básicas para llevar a cabo esa tarea, tales como podrían ser un motor de \glslink{renderizar}{renderizado} para visualizar gráficos en pantalla, detección de colisiones entre objetos, creación de animaciones, etc}}
\newglossaryentry{plataformas}{name={Plataformas}, text={plataformas}, description={Género de videojuegos en el que el personaje tendrá que hacer uso de mecánicas básicas como salto, sprint o escalada para ir sorteando obstáculos presentes en el mapa y superar el nivel}}
\newglossaryentry{renderizar}{name={Renderizar}, text={renderizar}, description={Proceso de generar un espacio que normalmente consta de tres dimensiones (3D), pero también se utiliza para generar espacios de dos (2D), representando en el proceso objetos, materiales, luces, etc}}
\newglossaryentry{scrum}{name={Metodología SCRUM}, text={SCRUM}, description={Metodología ágil orientada a equipos autogestionados que, muy resumidamente, se caracteriza por sus reuniones diarias de corta duración (15 minutos como máximo) para hablar de lo que se ha hecho, lo que se va a hacer y los problemas que se han encontrado y basada en sprints: la realización de un conjunto de funcionalidades (Sprint Backlog), extraídas del Product Backlog (el conjunto de funcionalidades totales)}}
\newglossaryentry{timing}{name={Timing}, text={timing}, description={Realizar una acción específica en el momento preciso. Normalmente está sujeto a la repetición de niveles, es decir, si el jugador realiza las mismas acciones en el mismo momento pasará el nivel o una parte del mismo (siempre que no haya elementos aleatorios involucrados)}}
\newglossaryentry{triplea}{name={Triple A (AAA)}, text={triple A}, description={Se denominan así a los videojuegos lanzados por grandes empresas, con un gran presupuesto detrás tanto para su producción como para su promoción. El motivo de esto es que las tres aes (AAA), tradicionalmente, significaban un diez en gráficos, un diez en sonido y un diez en \gls{jugabilidad}, pero el término ha ido evolucionando hasta llegar al descrito anteriormente, sin tener que ser necesariamente un juego de diez en ningún apartado}}