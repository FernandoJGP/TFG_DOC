%!TEX root =  tfg.tex
\chapter{Arranque}

\begin{quotation}[Novelist]{Ernest Hemingway (1899--1961)}
The good parts of a book may be only something a writer is lucky enough to overhear or it may be the wreck of his whole damn life -- and one is as good as the other.
\end{quotation}

\begin{abstract}
Durante este cap�tulo de la memoria se presentar� la lista de caracter�sticas que conforman el proyecto as� como el dise�o arquitect�nico, que describe los sistemas de producci�n, preproducci�n y pruebas.
\end{abstract}

\section{Lista de caracter�sticas}

TODO: Aplicar aqu� la primera iteraci�n de Feature Driven Development.

\section{Dise�o arquitect�nico}
\label{sec:arquitectura}
Para la realizaci�n del proyecto se dispone de dos dispositivos, un ordenador sobremesa y un ordenador port�til (a partir de ahora sobremesa y port�til) en el que se realizar�n las labores de producci�n, preproducci�n y pruebas.

\begin{table*}[h!]
	\centering
	\begin{coolTable}{ll}{2}
{Caracter�sticas sobremesa}
	\textbf{Sistema/s operativo/s}&Windows 10\\
	\textbf{Procesador}&Intel Core i7 930 2.80Ghz\\
	\textbf{Placa base}&Asus P6X58D-E\\
	\textbf{Memoria RAM}&12GB DDR3 1600Mhz PC3-12800 CL6 (2x4GB + 2x2GB)\\
	\textbf{Disp. de almacenamiento}&SSD 120GB + Disco duro 3TB SATA3 7200rpm\\
	\textbf{Tarjeta gr�fica}&Sapphire Radeon HD 7950 OC 3GB GDDR5\\
	\end{coolTable}
	\caption{Tabla de caracter�sticas del dispositivo sobremesa}
\end{table*}

\begin{table*}[h!]
	\centering
	\begin{coolTable}{ll}{2}
{Caracter�sticas port�til}
	\textbf{Sistema/s operativo/s}&Windows 10\\
	\textbf{Procesador}&Intel Core i7 4712MQ 2.3 GHz\\
	\textbf{Placa base}&(Dato no proporcionado)\\
	\textbf{Memoria RAM}&8GB DDR3 SODIMM (1x8GB)\\
	\textbf{Disp. de almacenamiento}&Disco duro 1TB SATA 5400rpm\\
	\textbf{Tarjeta gr�fica}&Nvidia GeForce GT820M 2GB GDDR3\\
	\end{coolTable}
	\caption{Tabla de caracter�sticas del dispositivo port�til}
\end{table*}