%!TEX root =  tfg.tex
\chapter{Primera iteraci�n}

\begin{quotation}[Novelist]{Ernest Hemingway (1899--1961)}
The good parts of a book may be only something a writer is lucky enough to overhear or it may be the wreck of his whole damn life -- and one is as good as the other.
\end{quotation}

\begin{abstract}
Resumen de lo que va a ocurrir en el cap�tulo. �Cu�l es el objetivo que tenemos con este cap�tulo?
\end{abstract}

\section{Caracter�sticas a desarrollar}

\begin{enumerate}
\item Funcionalidad A.
\item Funcionalidad B.
\end{enumerate}

\section{Dise�o}
Aqu� una discusi�n de c�mo va a afectar todo al dise�o

Debe insertarse un diagrama UML de dise�o con los cambios y hacer referencia en el texto as� Fig. \ref{fig:diseno01}.

\begin{figure}[htbp]
\begin{center}
\missingfigure{Aqu� el modelo de dise�o en formato vectorial preferentemente (pdf)}
% Incluir la figura quitando el comentario a la fila de abajo.
% \includegraphics[width=\textwidth]{myfile.pdf}
\caption{Diagrama UML de dise�o para la iteraci�n 1}
\label{fig:diseno01}
\end{center}
\end{figure}

Un memorando t�cnico por cada decisi�n de dise�o.

\begin{table*}[h!]
	\centering
	\begin{coolTable}{p{4cm}p{\textwidth-4.5cm}}{2}
{Memorando t�cnico 0001}
\textbf{Asunto}&�Cu�l es el problema?\\
\textbf{Resumen}&�Cu�l es la soluci�n propuesta?\\
\midrule
\textbf{Factores causantes}&Descripci�n pormenorizada del problema\\
\textbf{Soluci�n}&Descripci�n pormenorizada de la soluci�n propuesta\\
\textbf{Motivaci�n}&�Por qu� propone esta soluci�n?\\
\textbf{Cuestiones abiertas}& Factores a tener en cuenta en la soluci�n cuya dimensi�n se reconoce.\\
\textbf{Alternativas}&Otras soluciones consideradas y la raz�n por la que se excluyeron.\\
	\end{coolTable}
	\caption{Memorando t�cnico 0001}
\end{table*}


\section{Implementaci�n}

Un memorando t�cnico por cada decisi�n de implementaci�n y refactorizaci�n que afecte al dise�o.

\begin{asigResponsabilidad}{0001}{Prueba}
{[return\_type] method\_name1 (param1:type1, ...)}
\pasoPseudo{1. Paso 1.}
\pasoPseudo{2. Paso 2.}
\cabeceraMetodosBajoNivel
\pasoCodigo{1}{ClassName}{[return\_type] method\_name1 (param1:type1, ...)}{001}{SI}
\diagramaColaboracion{figures/colDiagram.png}
\end{asigResponsabilidad}

\begin{asigResponsabilidad}{alvotermar02}{Grubber}
{[return\_type] grubber (param1:type1, ...)}
\pasoPseudo{1. Lanzar 2 dados}
\pasoPseudo{2. Compara resultado de los dados con kicking del open-side}
\pasoPseudo{2.1. Si valor dados es menor o igual a kicking, avanza 10m}
\pasoPseudo{3.1. Si no hay defensa y el golpeo es exitoso, el pateador retiene la posesi\'on del bal\'on}
\pasoPseudo{3.2. Si hay defensa y el golpe es exitoso, el atacante tira un dado y suma su valor al de speed y strength y el defensor lanza 2 dados y lo suma al valor de speed y strength de su jugador, el vencedor ser\'a aquel que tenga m\'as puntos, si es igual, la posesi\'on es del defensor}
\pasoPseudo{4.1. Si no es exitoso y hay defensa el bal\'on pasa a posesi\'on del defensor}
\pasoPseudo{4.2. Si no es exitoso y no hay defensa de lanza un line-out}
\cabeceraMetodosBajoNivel
\pasoCodigo{1}{Dice}{[Integer] throwDice ()}{001}{SI}
\pasoCodigo{2}{ClassName}{[Int] compareKickingToDice (kicking:Integer, dice: Integer)}{001}{SI}
\pasoCodigo{2.1}{ClassName}{[Integer] setLine (line:Integer)}{001}{SI}
\pasoCodigo{4.2}{ClassName}{[Integer] lineOut ()}{001}{SI}
%\diagramaColaboracion{colDiagram.png}
\end{asigResponsabilidad}

\section{Pruebas}

Descripci�n de las pruebas realizadas al software

\section{Despliegue}

Breve resumen de c�mo se han desplegado los cambios en el sistema de producci�n.