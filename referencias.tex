%!TEX root =  tfg.tex
\chapter{Referencias}

\begin{abstract}
Este anexo tiene como objetivo recopilar la referencias usadas para la elaboraci�n del producto.
\end{abstract}

\section{Introducci�n}

A pesar de haber aprendido a utilizar Unreal Engine 4 de manera totalmente autodidacta, puesto que es una tecnolog�a que no se aprende durante el transcurso de la titulaci�n, no se ha necesitado de gran ayuda para el desarrollo del proyecto.

Parte de la culpa de que no se hayan consultado demasiadas referencias la tiene que se hayan realizado algunos peque�os prototipos, previamente al desarrollo del proyecto, usando \glspl{blueprint} puesto que, si bien al principio pueden ser una desventaja para alguien con conocimientos de programaci�n, son muy intuitivos de usar y realmente \textbf{nos tutelan} en todo momento (por ejemplo, cuando creamos un nodo y queremos configurar sus par�metros nos ofrece autom�ticamente una lista de elementos filtrados que nos podr�an interesar), haciendo el aprendizaje mucho m�s f�cil.

La gran mayor�a de las dudas que se han tenido elaborando el producto han sido peque�as dudas puntuales y, dejando de lado a �stas, tambi�n de �ndole matem�tica, puesto que los conocimientos matem�ticos para el desarrollo de un producto as� deben estar casi a la par que los de programaci�n.

\clearpage

\section{Referencias usadas}

Para la inmensa mayor�a de dudas puntuales, se ha consultado el centro de respuestas de "`Epic Games"' referente a "`Unreal Engine"', as� como los foros de "`Unreal Engine"' y la propia documentaci�n de �ste:
\begin{itemize}
	\item \url{https://answers.unrealengine.com/}
	\item \url{https://forums.unrealengine.com/}
	\item \url{https://docs.unrealengine.com/latest/INT/}
\end{itemize}

Pr�cticamente en estos tres sitios se ha concentrado la totalidad de la informaci�n consultada durante el proyecto.

Las otras principales referencias ya han sido mencionadas en el transcurso de la iteraciones, no obstante se volver�n a hacer aqu�.

Referencias usadas para la elaboraci�n del sistema de escalada:
\begin{itemize}
	\item \url{https://youtu.be/4yjcwZLQqlE}
	\item \url{https://youtu.be/H2xqW7lKkyw}
	\item \url{https://youtu.be/2vDjzr9EvUc}
	\item \url{https://youtu.be/fLLKgc0LDqc}
	\item \url{https://youtu.be/ABgcv21uivU}
\end{itemize}

Referencias usadas para la elaboraci�n de la inteligencia artificial de los "`vigilantes"':
\begin{itemize}
	\item \url{https://youtu.be/VAAHKNoIg0w}
	\item \url{https://youtu.be/UrSxEtOOHy8}
	\item \url{https://youtu.be/_Oll-Cl_JQw}
	\item \url{https://youtu.be/1TWNMkj4FP0}
	\item \url{https://youtu.be/sr9-w5jiUow}
\end{itemize}

Tambi�n podemos citar como referencia este tutorial que se us� parcialmente para la creaci�n del men� principal:
\begin{itemize}
	\item \url{https://youtu.be/SPzjL-j4Aas}
\end{itemize}

Ajeno a "`Unreal Engine"', tambi�n se utiliz� este tutorial para a�adir nodos ra�z a esqueletos que no lo ten�an:
\begin{itemize}
	\item \url{https://youtu.be/_QpDGNgg7IA}
\end{itemize}

Fuera del �mbito estricto de la programaci�n, tambi�n se ha usado esta gu�a para solventar los problemas de la iluminaci�n de escenarios:
\begin{itemize}
	\item \url{https://wiki.unrealengine.com/LightingTroubleshootingGuide}
\end{itemize}

Como nota importante, una vez que se complet� la implementaci�n del sistema de escalada, siguiendo el tutorial, nuestros conocimientos se multiplicaron: principalmente porque es un tutorial que toca pr�cticamente todos los aspectos de "`Unreal Engine"', pero tambi�n hubo que llevar a cabo modificaciones para que el sistema funcionara.

Estas son todas las referencias que podr�amos nombrar, ya que las peque�as (e innumerables) dudas son puntuales (o triviales) y no aportan demasiado al anexo.